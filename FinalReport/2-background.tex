\section{Background}

\subsection{Named Data Networking (NDN)}
Named Data Networking (NDN) \cite{zhang2014named} is a novel Internet architecture, which provides data-centric communication primitives. In NDN one Interest packet can only retrieve at most one Data packet. The Name defined in the Interest packet will be used as the routing information and also presented the content of the data. Since the NDN use the name to specify what it wants, the Interest can be satisfied in anywhere in the Network, which isolates the data from the location information. Interest packet could be satisfied by the original data producer or by a third-party storage provider. The in-network router cache can also serve the incoming Interest packets by looking up the Name in the Content Storage inside the router. Since in NDN the name presents the content of the Data packets, the data retrieving can be done by specify the name of the data instead of individual end-to-end connection. It is more easier to share data in distributed way in NDN instead of current TCP/IP network.

NDN uses a content-based authenticity model by requiring every data packets to be signed \cite{yu2015name}, which ensures that all data packets are generated by the trusted host. It can support the security requirement for the distribution data retrieving.

\subsection{Attribute Based Encryption(ABE)}
Attribute-based encryption is a type of public-key encryption \cite{brucker2010attribute}. Based on the attributes, the private key can be generated for different person and ciphertext can be generated by the attributes rule. Attribute-based encryption make it possible that the user with the set of attributes which satisfy the encryption attributes policy can decrypt the ciphertext. The concept of attribute-based encryption was first proposed by Amit Sahai and Brent Waters \cite{sahai2005fuzzy} and later by Vipul Goyal, Omkant Pandey, Amit Sahai and Brent Waters\cite{goyal2006attribute}. A crucial security aspect of Attribute-Based Encryption is collusion-resistance: An adversary that holds multiple keys should only be able to access data if at least one individual key grants access.

There are mainly two types of Attribute-Based Encryption schemes: Key-Policy Attribute-Based Encryption (KP-ABE)\cite{goyal2006attribute} and Ciphertext-Policy Attribute-Based Encryption (CP-ABE).\cite{bethencourt2007ciphertext} In KP-ABE, users' secret keys are generated based on an access tree that defines the privileges scope of the concerned user, and data are encrypted over a set of attribute. However, CP-ABE uses access trees to encrypt data and users' secret keys are generated over a set of attribute. In this paper the CP-ABE is applied.
