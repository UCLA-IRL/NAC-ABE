\section{Virtual Organization Over NDN}

The right side is the VO management system, which contains an \textbf{attribute authority} and a \textbf{VO manager}.
Attribute authority will setup the CPABE environment by generating public parameters and a secret master key.
The VO manager will create the VO, the data packet(s) containing the name category.
Since both UCLA and MIT produce data related to networking and number theory, there are two VOs containing networking lecture data names and number theory lecture data names respectively.
In this way, the authorised consumers could easily access the networking/number theory lecture data both from UCLA and MIT through the proper VO.

\subsection{VO Mangement System}

VO management system mainly contains two parts: attribute authority and VO manager.
The attribute authority will take the duty to setup the CPABE environment and issue decryption key for authorised consumers.
The VO manager will collect data information from each organization and produce VOs.

\subsection{Attribute Authority}
The attribute authority (AA) is a globally trust anchor.
All parties, no matter producer or consumer, should all trust the authority.
However, the existence of attribute authority won't lower down the security of the whole system and attribute authority won't become the single point of failure when system is running.
In the VO system, the attribute authority only works at the preliminary phase.
Once finish the system setup, the attribute authority should go offline to avoid potential attack from outside.

Before the system start running, AA would generate the public parameters which is a data packet and a secret master key.
In our system, the public parameter has the naming convention \ndnName{/[AA-prefix]/PUB/<timestamp>}.
For instance, in the example, a possible name can be \ndnName{/vo/authority/PUB/201706151200}.
The public parameters data packet will be signed by VO management system's certificate.
Regarding the master key, once the secret master key get compromised, the system will be broken.
In the preliminary phase, the authority will also issue consumer's decryption key based on the information provided by the token issuer.
The detail will be illustrated in the section \ref{token-issuer}.
This makes sure that the AA will only talk with trusted parties during the preliminary phase.

Once the system starts running, the AA should go offline immediately.
In the future, the AA should be awaken only when consumer's key get compromised and when there are new attributes.
Notice, even when AA is working, AA only needs to talk with token issuer, which is trusted.
If the token issuer corrupts, there is no security in system; thus only trustworthy party, like universities, government departement and etc., can become token issuer.

\subsection{VO Manager}
VO manager is the service provider who define the name category and produce VO packets.
All the organizations who wants to join the VO are supposed to provide the data information for VO manager.
The information should at least contains the data name and a category suggestion.
VO manager can use the application-readable data name to put each name into one or more VOs.

To protect the category information from unauthorized users, the data content should be encrypted by specific attribute.
In our system, there are VO attributes: "vo-001", "vo-002" and etc.
This kind of attribute is used to encrypt the name category.
Following the naming convention of data packet, the name of the VO packet should be like: \ndnName{/vo-manager/networking-vo/vo-001}.
The last component is the attribute policy, which means the single attribute vo-001 is enough to decrypt the category.
Only user who has the corresponding key could see the category.
Notice that even users can see the category, it does not mean that users could consume the data for the data packets are encrypted by organization's policies.
Assume one data name from network lecture VO's data name is \ndnName{/ucla/network/lesson1/VO-001 OR (ucla and professor)}, it means all the consumer who can see the name category can consume the data packet.
However, if the data name is \ndnName{/ucla/network/lesson2/ucla and professor}, it means the consumer could only know the existence of the data but cannot consume the content.

Just like other data packet, the VO packet must be signed by the VO management system.
This allow the VO packet to be widely cached and thus lower down the latency.
