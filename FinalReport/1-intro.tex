\section{Introduction}

Resource sharing across the Internet has always been a need that is widely recognized for decades.
Yet, the isolation is still there and becomes a roadblock.
A large amount of efforts have been made to achieve collaboration and break the isolation.
However, it is still a big challenge to ensure the privacy, security, efficiency and semantic content at the same time.

Today, most resource sharing among multiple organizations, i.e. service provider, data producer, is done through the direct communication between two parties.
There is rarely or no collaboration platform that organize the resources from multiple organizations.
Also, the security and privacy of current resource sharing are usually ensured by following two mechanisms:
\begin{itemize}
\item \textbf{Protecting the communication channel} \\
Protocol like TLS encrypts the end-to-end channel to provide data exchange.
However, protecting channel always means to sacrifice the efficiency of content distribution.
For example, content multicast is not supported by channel-based security.
\item \textbf{Protecting the content} \\
There are mechanisms that provide encryption-based security.
For example, there are encryption-based access control systems using traditional asymmetric public key encryption or attribute-based encryption.
The content-based encryption allows the encrypted data to be stored publicly.
However, protecting content always means the lost of the semantic meaning.
For example, the resource sharing platform have no clue of the content and it's hard to organize millions of resources with blind eyes.
\end{itemize}
