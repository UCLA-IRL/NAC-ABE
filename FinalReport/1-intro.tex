\section{Introduction}

Resource sharing across the Internet has always been a need that is widely recognized for decades.
Yet, the isolation is still there and becomes a roadblock.
A large amount of efforts have been made to achieve collaboration and break the isolation.
However, it is still a big challenge to ensure the privacy, security, efficiency and semantic content at the same time.

Today, most resource sharing among multiple organizations, i.e. service provider, data producer, is done through the direct communication between two parties.
There is rarely or no collaboration platform that is able to organize the resources from multiple organizations.
Also, the security and privacy of current resource sharing are usually ensured by following two mechanisms:
\begin{itemize}
\item \textbf{Protecting the communication channel} \\
Protocol like TLS encrypts the end-to-end channel to provide data exchange.
However, protecting channel always means to sacrifice the efficiency of content distribution.
For example, content multicast is not supported by channel-based security.
\item \textbf{Protecting the content} \\
There are mechanisms that provide encryption-based security.
For example, there are encryption-based access control systems using traditional asymmetric public key encryption or attribute-based encryption.
The content-based encryption allows the encrypted data to be stored publicly.
However, protecting content always means the lost of the semantic meaning.
For example, the resource sharing platform have no clue of the content and it's hard to organize millions of resources with blind eyes.
\end{itemize}

To break the roadblock of the resource sharing, we proposed and designed Virtual Organization (VO), a name category service over Named Data Networking (NDN).
NDN has fundmental different with TCP/IP network: instead of delivering packets between two IP addresses, hosts in NDN fetch content by the content name.
Hosts send out \textit{interest packet} which contains the target content name, and the interest would fetch \textit{data packet} which contains the content.
NDN provides the content-centric security in network layer by requiring each data packet to be signed by the data producer, so that the data consumer doesn't need to care where the data is from.
We defer the detail to background section.

We also implement the ciphertext-policy attribute-based encryption (CPABE) in our system.
CPABE use the attributes and attribute policy to realize the access control.
The decryption key is bound with a set of attributes, each attribute is a string.
The ciphertext is generated by encrypting plaintext by attribute policy, which is also a string.
Only users with sufficient attributes could decrypt the content.
For instance, a producer could encrypt the content by the attribute policy: (UCLA or MIT) and Professor.
In this way, only professors from UCLA and MIT are able to consume the data.
The detail of the CPABE is introduced in background section.

In this paper, we use the term \textbf{Organization} to denote the data producer (service is also one type of data) and use the term \textbf{Virtual Organization} to represent one category which is like a real organization from consumer's view.
In the system, the organizations would produce the data and do the encryption using attribute-based encryption (ABE).
The organizations also need to notice the VO management system the list of produced data names.
With the list of data names produced by each organization, the VO management system can take use of the semantic data names to define different VOs and thus each VO contains a list of data names.
Notice that the VO is also encrypted by VO related attributes, which is usually the name of the VO identifier.
The VO system has some desirable features.
\begin{itemize}
\item \textbf{Category-based Resource Sharing} \\
Every VO is a category containing a list of data names organized by the VO management system.
All the name in one VO are supposed to have some inner connection, i.e. semantically related.
For example, all the names could be related to political news.
Therefore, VO is considered as a indirection between data consumer and organizations.
Since the data name is application-readable or even human-readable, VO management system can easily category the data from different organizations.
One benefit of the design is that usually one or several NDN data packet(s) could hold one complete category.
The VO data packet will be signed by the VO management system and therefore the packet could be cached anywhere.
Even the VO is fetched from the untrusted source, the VO could be trust if the signature is valid.
\item \textbf{High Efficiency: Multicast and Content-centric} \\
As mentioned, usually one (or several) packet is enough to hold a complete VO.
Even though there is a large amount of interest packets requesting the same VO, all the interests can be merged and eventually there may be only one interest packet; thus the VO management system can only reply for only one time.
Notice the data packet is signed by VO management system, which means the source of the data packet doesn't matter.
Regardless of where the data is fetched from, once the crypto signature can be verified, the data packet can be trusted.
As for the stale VOs, consumers could decide whether the VO is still fresh by checking the signature timestamp and fresh period.
\item \textbf{Fine-grained Access Control} \\
CPABE allows each party in the system to become a producer immediately: one can use the plaintext attribute policy as the public key and do the encryption.
Also, the access policy is a set of attribute strings combined by logic gates like \textit{AND}, \textit{OR} and \textit{LargerThan}, so it is easy for organization to define the policy.
CPABE also decouples the data encryption and the knowledge of consumer: producer does not necessarily know the detail of consumers, producer only needs to care the attribute of the target consumers.
\end{itemize}