\section{Conclusion}

In this paper, we proposed the Attribute-based Access Control over NDN which achieves the secure access control protocol over the NDN. The proposed ABAC-NDN includes the following advanced features:

\begin{itemize}
	\item Semi-distributed system.
	\item Asynchronous data production and consumption.
	\item Decoupled access control and knowledge of the consumers.
	\item Allow data owner and producer in different devices.
	\item Allow attribute authority and token issuer in different devices.
	\item Any party could become a producer immediately.
\end{itemize}

Compare to other encryption method in NDN, the ABAC is more feasible and more flexible. The attribute based encryption can serve NDN application in a good way, since the build-in distribution support in NDN can let consumer fetch the content based on Name and decrypt with attributes without other information like location in TCP/IP network. By encrypting and decrypting content based on the attributes, it can achieve the virtual organization concept we proposed in this paper.

With above advance features, we implemented the ABAC-NDN in a C++ library. The library can support the ABAC based application in NDN in an easy way. For now the unit tests and integrated test can verify the correctness of the proposed ABAC-NDN. It shows that our code base can successfully achieve the concept of virtual organization based on the attributes.